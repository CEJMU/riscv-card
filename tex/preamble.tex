\usepackage[
    left=1cm,
    right=1cm,
    top=3cm,
    bottom=1.5cm,
    includefoot
    ]{geometry}  
\usepackage[T1]{fontenc}
\usepackage{fetamont}
\usepackage{inconsolata}
\usepackage{array}
\usepackage{titling}
\usepackage{graphics}
\usepackage{multirow}
\usepackage{fancyhdr}
\usepackage{hyperref}
\usepackage[no-math]{fontspec}
\usepackage{tabularray}
\usepackage[
    type={CC},
    modifier={by},
    version={4.0},
]{doclicense}

\graphicspath{img}

\usepackage{background}
\backgroundsetup{scale = 1, angle = 0, opacity = 1.0, contents = {\includegraphics[width=\paperwidth, height=\paperheight, keepaspectratio]{img/bg_en}}}

% Schriften:
\defaultfontfeatures[FFMetaPro]
{
    Extension = .TTF,
    UprightFont = FFMetaProLight,
    ItalicFont = FFMetaProLightIt,
    BoldFont = FFMetaProBold,
    BoldItalicFont = FFMetaProBoldIt
}
\setsansfont{FFMetaPro}
\renewcommand{\familydefault}{\sfdefault} 


% New column types to use in tabular environment for instruction formats.
% Allocate 0.18in per bit.
% \newcolumntype{I}{>{\centering\arraybackslash}p{0.18in}}
% Two-bit centered column.
\newcolumntype{W}{>{\centering\arraybackslash}p{0.36in}}
% Three-bit centered column.
\newcolumntype{F}{>{\centering\arraybackslash}p{0.54in}}
% Four-bit centered column.
% \newcolumntype{Y}{>{\centering\arraybackslash}p{0.72in}}
% Five-bit centered column.
\newcolumntype{R}{>{\centering\arraybackslash}p{0.9in}}
% Six-bit centered column.
% \newcolumntype{S}{>{\centering\arraybackslash}p{1.08in}}
% Seven-bit centered column.
\newcolumntype{O}{>{\centering\arraybackslash}p{1.26in}}
% Eight-bit centered column.
% \newcolumntype{E}{>{\centering\arraybackslash}p{1.44in}}
% Ten-bit centered column.
% \newcolumntype{T}{>{\centering\arraybackslash}p{1.8in}}
% Twelve-bit centered column.
% \newcolumntype{M}{>{\centering\arraybackslash}p{2.2in}}
% Sixteen-bit centered column.
% \newcolumntype{K}{>{\centering\arraybackslash}p{2.88in}}
% Twenty-bit centered column.
% \newcolumntype{U}{>{\centering\arraybackslash}p{3.6in}}
% Twenty-bit centered column.
% \newcolumntype{L}{>{\centering\arraybackslash}p{3.6in}}
% Twenty-five-bit centered column.
% \newcolumntype{J}{>{\centering\arraybackslash}p{4.5in}}

\renewcommand{\familydefault}{bch}

% Typewriter font 
\newcolumntype{T}{>{\texttt\bgroup}l<{\egroup}}
% Typewriter font (code) 
\newcolumntype{C}{>{\texttt\bgroup}p{2.5in}<{\egroup}}

\newcommand{\instbit}[1]{\mbox{\scriptsize #1}}
\newcommand{\instbitrange}[2]{~\instbit{#1} \hfill \instbit{#2}~}

%% Header / Footer %%

\newcommand{\headerfont}{\fontfamily{lmr}\selectfont}
%\pagestyle{fancy}
%\headheight=24pt
%\footskip=24pt
%\lhead{\headerfont RISC-V Reference Card}
%\rhead{\headerfont V0.2}
%\rfoot{CE JMU 2024}

%% Title %%
\pretitle{\vspace{-0.75in}\begin{center} \huge \sffamily \bfseries}
\title{RISC-V Reference}
\posttitle{\end{center}}
\preauthor{\begin{center} \sffamily}
\author{Matthias Jung <\href{mailto:m.jung@uni-wuerzburg.de}{m.jung@uni-wuerzburg.de}>}
\postauthor{\end{center}}
\predate{}
\date{}
\postdate{}
